%%%%%%%%%%%%%%%%%%%%%%%%%%%%%%%%%%%%%%%%%%%%%%%%%%%%%%%%%%%%%%%%%%%%%%%%
\chapter{Requisitos}
\label{ch:requisitos}
%\fxnote{En este capítulo, bla bla bla}
En este capítulo se presentan los requisitos a cumplir por la aplicación. Comenzaremos explicando el escenario al que pertenece la aplicación y terminaremos presentando las historias de uso de la misma.
\section{Escenario}
%\fxnote{¿Qué signfica ``escenario'' aquí? La palabra escenario tiene un significado muy claro en la ingeniería de requisitos y suelen ser descripciones de situaciones de uso de la aplicación.}

El escenario que se pretende %\fxnote{corrección ortográfica?} 
cubrir con la aplicación es la posibilidad de navegar y realizar apuestas sobre eventos deportivos del portal Betfair.com. Se pretende cubrir las funcionalidades básicas que ofrece Betfair en su portal web para poder realizar apuestas. %\fxnote{Cuidado, las funcionalidades son importantes a través de la web, nosotros no las hemos implementado todas.}

%\section{Historias de usuario}

%\fxnote{Otra idea puede ser referirnos al interfaz web para dejar claras algunos requisitos.}

%\fxnote{Entre los requisitos llamados habitualmente ``no funcionales'' (odio el adjetivo) está el de adaptar la forma en la que se muestra la información a las limitaciones y capacidades del propio terminal.}

%\fxnote{¿Qué vas a contar aquí? ¿Qué significa ``historias de uso''? ¿No será ``historias de usuario''? ¿Vas a seguir algún ``estándar''? ¿XP? Como ejemplo: en \url{http://en.wikipedia.org/wiki/User_story} puedes encontrar algo de texto introductorio, incluidas ventajas e inconvenientes. En \url{http://www.agilemodeling.com/artifacts/userStory.htm} puedes encontrar más detalle sobre cómo hacerlas. En el TFC de Antonio puedes encontrar incluso macros latex para que aparezcan más bonitas.}

%\fxnote{Entre los requisitos puedes añadir seguir alguna guía de estilo de Apple o del desarrollo para iPhone (que no se si existen).}

\section{Historias de usuario}
%\fxnote{Mejor ``Historias de Uso''}
\subsection{Instalación} Para poder instalar la aplicación sólo se necesitará una cuenta del programa iTunes Store de Apple. Es gratuita. La aplicación estará disponible dentro del programa App Store del dispositivo. Solamente habrá que descargar la aplicación de dicho portal. Alternativamente podemos descargarla también desde el programa iTunes para Windows y Mac. Una vez descargada podemos transferirla al dispositivo sincronizando mediante Wifi o cable USB. %\fxnote{¿No existen otras forma de instalar las aplicaciones?}
\subsection{Actualizar la aplicación}
La aplicación  App Store  del dispositivo será la encargada de comunicar al usuario la aparición de una nueva versión del aplicativo. Para actualizarla simplemente habrá que seguir las indicaciones de dicho programa. %\fxnote{Lo mismo  de antes}

\subsection{Desinstalar}
Para desinstalar la aplicación simplemente se mantiene pulsado el icono de la misma unos segundos hasta que el icono empiece a vibrar. Inmediatamente pulsamos sobre el icono X que aparecerá es la esquina superior izquierda del icono de la aplicación. %\fxnote{¿qué es la ``X''? ¿es estándar?} que aparecerá sobre la misma.
\subsection{Ejecución}
Para lanzar la aplicación sólo hay que pulsar el icono que aparece en la pantalla principal del dispositivo.
\subsection{Configurar la aplicación}
Se podrán configurar los diferentes aspectos de la aplicación tales como el idioma o las credenciales de usuario a través del menú de ajustes del dispositivo. En el caso de ser la primera vez que se lance la aplicación, ésta automáticamente redirigirá al usuario a dicha pantalla para configurarla.
\subsection{Gestión de los eventos}
Se podrá navegar y obtener información de todos los eventos deportivos disponibles del portal Betfair a través una  %\fxnote{más bien de ``una''}
 jerarquía de menús de la aplicación siguiendo las hojas de estilo de interfaz de usuario de Apple.% \fxnote{¿Qué es un evento?}
\subsection{Realizar una apuesta}
%\fxnote{Recuerda que esto son requisitos, no el manual de usuario. Debemos decir lo que la aplicación ``va'' a permitir o lo que esperamos de ella. El ``cómo'' está más adelante.}
La aplicación permitirá la realización de una apuesta sobre un evento específico incluyendo la cantidad y cuota deseada. El sistema notificará al usuario el resultado de la operación.
\subsection{Gestionar las apuestas}
La aplicación será capaz de recopilar todas las apuestas realizadas sobre Betfair. Por cada apuesta, el sistema mostrará las opciones disponibles: información detallada de la apuesta, asesoramiento para trading y estado actual del mercado. %\fxnote{¿Que son?}
\subsection{Realizar trading sobre una apuesta ya realizada}
El sistema será capaz de asesorar para realizar trading sobre una apuesta ya realizada anteriormente. Para ello, se mostrará una tabla con las opciones disponibles dentro del resumen de una apuesta ya realizada. Se podrán configurar los tradings para que se lancen automáticamente por parte de la aplicación.

%\fxnote{Aunque no acabemos implementando, el asesoramiento para trading automático debería estar en la lista de requisitos.}

%%% Local Variables: 
%%% mode: latex
%%% TeX-master: "tfc-betfair-ios"
%%% TeX-PDF-mode: t
%%% ispell-local-dictionary: "castellano"
%%% End: 
